
\usepackage[colorlinks,linkcolor=靛青]{hyperref}% 影響扉䈎
\newcommand*\陰文[1]{\tikz[baseline=(char.base)]{
			\node[shape=rectangle,fill=black!90,draw,inner sep=0pt] (char) {\color{white}#1};}}
\setCJKmonofont[RawFeature={vertical:+vert}]{仿宋} % 等寬 for \texttt
\setCJKsansfont[RawFeature={vertical:+vert}]{Noto Sans CJK KR} % 无襯綫 for \textsf %[Vertical=RotatedGlyphs]
\newCJKfontfamily[fangsung]\fangsung[RawFeature={vertical:+vert}, ItalicFont=文悦古体仿宋 (非商业用途)]{FangSong}
\newCJKfontfamily[puncts]\puncts[RawFeature={vertical:+vert}, ItalicFont=標點符號-R]{標點符號-R}
\xeCJKsetup{PunctFamily=puncts, CJKecglue=\null}% 標點字型fangsung,漢字與西文、符號之間隔
\setmainfont{仿宋}%Times New Roman
\xeCJKsetup{
	AutoFallBack=true,  %開啓字體回退,即字體遇到不能顯示的字符時,使用後備字體顯示
}
\setCJKfallbackfamilyfont{\CJKrmdefault}{{HYFangSong_GB18030Super-ExtB},{KaiXinSongB}}

\usepackage[backend=biber,style=ieee,citestyle=numeric,giveninits=true,sorting=none,maxbibnames=99,dashed=false,doi=false]{biblatex}
\makeatletter
\clubpenalty0
\interlinepenalty=0
\@clubpenalty \clubpenalty
\widowpenalty0
\makeatother
\addbibresource{迩原參攷文獻.bib}
\AtBeginBibliography{\vspace*{-10pt}}
\setlength\bibitemsep{0pt}
\renewcommand*{\newunitpunct}{ }
\renewcommand\multicitedelim{、}
\renewcommand*{\citesetup}{%
	\sffamily\fontsize{0.5em}{0}\selectfont
	\biburlsetup
	\frenchspacing}
\DeclareCiteCommand{\cite}{〔}%\usebibmacro{prenote}
{\usebibmacro{citeindex}%
	\printtext[bibhyperref]{%
		\printfield{prefixnumber}%
		\rotatebox{90}{\kern-.5em\fontspec{筭數}[RawFeature={+ss01,+fwid}]{\printfield{labelnumber}}}%
		\ifbool{bbx:subentry}
		{\printfield{entrysetcount}}
		{}}%
}
{\multicitedelim}%
{〕}
\DeclareFieldFormat[article,incollection,unpublished,book,inbook]{title}{\fangsung#1}
\DeclareFieldFormat{labelnumberwidth}{\rotatebox{90}{\kern-.5em\fontspec{筭數}[RawFeature={+ss01,+fwid}]{#1}}}
\DeclareFieldFormat[article,incollection,unpublished,book,inbook]{volume}{#1}
\DeclareFieldFormat{pages}{\raisebox{-.37em}{\fontspec{筭數}{#1}}}
\DeclareFieldFormat{date}{\raisebox{-.3em}{#1}}
\DeclareFieldFormat[inbook]{chapter}{\fangsung·#1}
\DeclareDelimFormat{multinamedelim}{、}
\DeclareDelimFormat{finalnamedelim}{、}
%\bibrangedash
\DeclareBibliographyDriver{book}{%
	\printnames{author}%
	\newunit\newblock
	\printfield{title}%
	\newunit\newblock
	\printlist{publisher}%
	\newunit
	\printlist{location}%
	\newunit\newblock
	\rotatebox{90}{\kern-1.5em\fontspec{筭數}[RawFeature={+ss01}]{\printfield{year}}}年\printfield{month}%
	\newunit\newblock\printfield{pages}
}
\DeclareBibliographyDriver{article}{%
	\printnames{author}%
	\newunit\newblock
	\printfield{title}%
	\newunit\newblock
	\printfield{journaltitle}%
	{\fangsung\printfield{volume}}%
	\newunit\newblock\quad
	\rotatebox{90}{\kern-1.5em\fontspec{筭數}[RawFeature={+ss01}]{\printfield{year}}}年\printfield{month}%
	{\fangsung 䈎第}\printfield{pages}
	%\finentry
}
\DeclareBibliographyDriver{inbook}{%
	\printnames{author}%
	\newunit\newblock
	\printfield{title}%
	\printfield{chapter}
	\newunit\newblock
	\printlist{publisher}%
	\newunit
	\printlist{location}%
	\newunit\newblock
	\rotatebox{90}{\kern-1.5em\fontspec{筭數}[RawFeature={+ss01}]{\printfield{year}}}年\printfield{month}%
	{\fangsung 䈎第}\printfield{pages}
}
\DefineBibliographyStrings{english}{
	january = \rotatebox{90}{\kern-.5em\fontspec{筭數}[RawFeature={+ss01,+fwid}]{1}}月,
	february = \rotatebox{90}{\kern-.5em\fontspec{筭數}[RawFeature={+ss01,+fwid}]{2}}月,
	march = \rotatebox{90}{\kern-.5em\fontspec{筭數}[RawFeature={+ss01,+fwid}]{3}}月,
	april = \rotatebox{90}{\kern-.5em\fontspec{筭數}[RawFeature={+ss01,+fwid}]{4}}月,
	may =\rotatebox{90}{\kern-.5em\fontspec{筭數}[RawFeature={+ss01,+fwid}]{5}}月,
	june = \rotatebox{90}{\kern-.5em\fontspec{筭數}[RawFeature={+ss01,+fwid}]{6}}月,
	july = \rotatebox{90}{\kern-.5em\fontspec{筭數}[RawFeature={+ss01,+fwid}]{7}}月,
	august = \rotatebox{90}{\kern-.5em\fontspec{筭數}[RawFeature={+ss01,+fwid}]{8}}月,
	september = \rotatebox{90}{\kern-.5em\fontspec{筭數}[RawFeature={+ss01,+fwid}]{9}}月,
	october = \rotatebox{90}{\kern-.2em\fontspec{筭數}[RawFeature={+ss01}]{10}}月,
	november = \rotatebox{90}{\kern-.2em\fontspec{筭數}[RawFeature={+ss01}]{11}}月,
	december = \rotatebox{90}{\kern-.2em\fontspec{筭數}[RawFeature={+ss01}]{12}}月,
}